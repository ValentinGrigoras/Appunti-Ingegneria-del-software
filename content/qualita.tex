\section{Fornire una definizione del concetto di \textit{qualità}, applicabile al dominio dell'ingegneria del software. Discutere concisamente quali attivita' il proprio gruppo di progetto didattico abbia svolto nella direzione di tale definizione, indicando allo stato attuale di progetto  i migliori e i peggiori risultati ottenuti, offrendo una spiegazione dell'esito}

La \textbf{qualità di un oggetto} è una caratteristica che si basa su proprieta' misurabili del prodotto, cioè su quantità confrontabili con degli standard prefissati; nel caso del software, però, queste proprietà \textit{misurabili} sono più difficili da quantificare rispetto agli oggetti fisici. Tuttavia, anche per il software sono state standardizzate delle metriche che riguardano la complessita' ciclomatica, la coesione, il numero di function-points, il numero di righe di codice.

Il \textbf{controllo della qualità} viene fatto attraverso l'attività di software quality assurance che si compone di un’attività di gestione della qualità, revisioni tecniche formali svolte durante il processo, una strategia di collaudi su più livelli, una gestione della documentazione e delle modifiche, una procedura che garantisca la conformità allo standard dello sviluppo, e infine meccanismi di misurazione e stesura dei resoconti.

La \textbf{qualità del software} è il rispetto dei requisiti funzionali e prestazionali enunciati esplicitamente, la conformità a standard di sviluppo esplicitamente documentati e le caratteristiche implicite che ci si aspetta da un prodotto software realizzato professionalmente.

Da questa definizione emergono tre punti fondamentali per lo svolgimento dell'attività di SQA:
\begin{enumerate}
 \item i requisiti sono alla base delle misurazioni della qualità; la non conformità ai requisiti implica mancanza di qualità;
\item gli standard specificati definiscono i criteri da seguire durante lo sviluppo del software;
\item anche i requisiti impliciti devono essere tenuti in cosiderazione; un software che rispetta i requisiti
espliciti ma non quelli impliciti è spesso un software di scarsa qualità.
\end{enumerate}

Durante il corso del progetto, la qualità è stata perseguita definendo obiettivi ad essalegati, monitorati mediante misurazioni metriche pertinenti e raggiunti tramite apposite strategie. L'attività di quality assurance ha accertato che il grado di conseguimento degli obiettivi fosse in linea con quanto previsto, e l'impianto amministrativo, le procedure e gli strumenti automatici dedicati hanno concretizzato la politica di qualità stabilita dal gruppo. Sotto uno sguardo critico, lo svolgimento di tali attività è stato tuttavia superficiale e lacunico, producendo falle nel sistema di attuazione della qualità cui ha conseguito il mancato raggiungimentodi alcuni obiettivi proposti. Tale problema si sarebbe probabilmente potuto mitigare con una migliore attività di formazione del personale.