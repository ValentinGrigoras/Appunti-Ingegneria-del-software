\section{Descrivere la tecnica di classificazione e tracciamento dei "requisiti" adottata nel proprio progetto didattico e discuterne l'efficacia e i limiti eventualmente riscontrati.}

Un \textbf{requisito} è una descrizione astratta di come dovrebbe essere il comportamento del sistema o di alcuni suoi vincoli. Il documento analisi dei requisiti ha lo scopo di elencare e descrivere in modo formale l'insieme dei requisiti.

Gestire la \textbf{tracciabilità dei requisiti} nel flusso di progetto è la chiave per una progettazione più efficiente e più efficace.  Più efficiente perché consente di velocizzare e gestire in modo ottimale tutte le varianti di progetto, in ogni sua fase. Inoltre è più efficace, perché permette di mantenere il progetto fedele alle specifiche iniziali, che corrispondono ai requisiti desiderati del committente, anche se lungo il flusso emergono vincoli non previsti inizialmente.

L'attività di \textbf{classificazione e tracciamento dei requisiti} si è divisa in più fasi. Inanzitutto si è scelto il modello per il ciclo di vita del software più idoneo al nostro progetto. Si è scelto un approccio incrementale perchè i requisiti non erano sufficientemente chiari nella prima fase del progetto. Questo ci è permesso, in seguito alla discussione con il proponente, di redigere i requisiti obbligatori e quelli desiderabili, anche se col avanzare del tempo si è dovuto rinegoziare con il proponente cambiando alcuni requisiti.

Per la \textbf{scoperta dei requisiti} sono state utilizzate alcune tecniche, quali:
\begin{itemize}
\item identificazione dei \textbf{casi d'uso}, che ci è permesso di analizzare le modalità di utilizzo del sistema.
\item \textbf{interviste} con il proponente;
\item \textbf{brainstorming}, grazie alla quale sono state raccolte e organizzate le idee di ogni componente del gruppo su come il sisteme deve comportarsi;
\end{itemize}

La tecnica utilizzata per la traccibilità dei requisiti è stata quella di assegnare un identificatore univoco a ciascun requisito rigorosamente documentato nel documento norme di progetto. L'identificatore ha la seguente forma:
\begin{center}
\textbf{R[Priorità][Tipo][Codice]}
\end{center}
Ogni requisito ha una sua priorità che può essere di tre livelli: obbigatorio (0), desiderabile (1) e opzionale (2).

Ogni requisito si differenzia in 4 diverse tipologie: funzionali (F), prestazionali (P), qualitativi(Q).

Apposite tabelle hanno poi permesso di associare ogni requisito alle proprie fonti e ai casi d’uso corrispondenti. 

Con queste premesse, non sono emersi particolari problemi nel tracciamento dei requisiti in sé, quanto più nel livello di dettaglio degli stessi che talvolta rendeva ambigua la conferma del loro soddisfacimento.