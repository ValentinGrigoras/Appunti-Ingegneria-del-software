\section{Proxy}
\subsection{Descrizione}
Si tratta di un pattern strutturale basato su oggetti che viene utilizzato per accedere ad un’oggetto complesso tramite un oggetto semplice.
Questo pattern può risultare utile se l’oggetto complesso:
\begin{itemize}
\item richiede molte risorse computazionali;
\item richiede molto tempo per caricarsi;
\item è presente su una macchina remota e il traffico di rete determina latenze ed overhead;
\item non definisce delle policy di sicurezze e consente un accesso indiscriminato;
\item non viene mantenuto in cache ma viene rigenerato ad ogni richiesta.
\end{itemize}

\subsection{Applicabilità}
\begin{itemize}
	\item \textbf{Remote Proxy}: rappresentazione locale di un oggetto che si trova in uno spazio di indirizzi differenti;
	\item \textbf{Virtual Proxy}: gestisce la creazione su richiesta di oggetti costosi;
	\item \textbf{Protection Proxy}: controlla l'accesso all'oggetto originale. Questo tipo di proxy si rivela utile quando possono essere definiti diritti di accesso diversi per gli oggetti;
	\item \textbf{Riferimento intelligente}: sostituisce un puntatore puro a un oggetto consentendo l'esecuzione di attività aggiunte quando si accede all'oggetto referenzito;
\end{itemize}