\section{Proxy}
\subsection{Descrizione}
Si tratta di un pattern strutturale basato su oggetti che viene utilizzato per accedere ad un’oggetto complesso tramite un oggetto semplice.
Questo pattern può risultare utile se l’oggetto complesso:
\begin{itemize}
\item richiede molte risorse computazionali;
\item richiede molto tempo per caricarsi;
\item è presente su una macchina remota e il traffico di rete determina latenze ed overhead;
\item non definisce delle policy di sicurezze e consente un accesso indiscriminato;
\item non viene mantenuto in cache ma viene rigenerato ad ogni richiesta.
\end{itemize}

\subsection{Applicabilità}
\begin{itemize}
	\item \textbf{Remote Proxy}: rappresentazione locale di un oggetto che si trova in uno spazio di indirizzi differenti. Tipicamente permette di accedere a risorse distribuite sulla rete come se fossero accessibili come oggetto locale;
	\item \textbf{Virtual Proxy}: gestisce la creazione su richiesta di oggetti costosi;
	\item \textbf{Protection Proxy}: controlla l'accesso all'oggetto originale. Questo tipo di proxy si rivela utile quando possono essere definiti diritti di accesso diversi per gli oggetti;
	\item \textbf{Riferimento intelligente}: sostituisce un puntatore puro a un oggetto consentendo l'esecuzione di attività aggiunte quando si accede all'oggetto referenzito;
\end{itemize}

\subsection{Struttura}
\begin{figure}[H]
\centering
\includegraphics[scale=0.6]{images/proxy1}
\caption{Struttura proxy\label{fig:UC3}}
\end{figure}


\begin{itemize}
	\item \textbf{Subject}
	\begin{itemize}
		\item La classe Subject è una classe astratta ed è la classe base del Proxy e del RealSubject;
		\item Definisce i membri che verrano implementati dalle sottoclassi;
	\end{itemize}
	\item \textbf{RealSubject}
	\begin{itemize}
		\item La classe RealSubject è la classe complessa che desideriamo utilizzare in modo efficiente, senza tanto sprecco di risorse;
	\end{itemize}
		\item \textbf{Proxy}
	\begin{itemize}
		\item Gli oggetti Proxy contengono un'istanza privata di un oggetto RealSubject;
		\item Gli oggetti Client eseguono azioni sul proxy che vengono passati all'oggetto RealSubject;
		\item I riusltati dei membri di RealSubject vengono restituiti al client tramite il Proxy.
	\end{itemize}
	
\end{itemize}

Nel diagramma delle classi vediamo come il client dipende solamente dall'interfaccia. Invece dell'oggetto reale, il client potrebbe utilizzare il proxy. Quando l'oggetto proxy viene chiamato fa le sue cose e infine inoltra la chiamata all'oggetto reale.

\subsection{Parole chiavi}
\begin{itemize}
\item \textit{remote proxy}: ambasciatore, governare, remoto, oggetti dislocati nella rete;
\item \textit{virtual proxy}: risorse, locale;
\item \textit{puntatore intelligente}: riferimento, locale;
\end{itemize}

\subsection{Associazione}
\begin{itemize}
\item viene spesso associato con un pattern proxy o adapter o strategy;
\end{itemize}

