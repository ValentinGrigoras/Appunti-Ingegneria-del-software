\subsection{Discutere la differenza tra le attività di "versionamento" e "configurazione" come vengono applicate all'ambito dello sviluppo sw}
Una \textbf{configurazione} indica le parti di un prodotto software e come esse vengono messe insieme. Le attività di configurazione vanno pianificate e la loro gestione va automatizzata. Esse servono a mettere in sicurezza la baseline, prevenire sovrascritture e permettere il ritorno alle versioni precedenti.
Nell'ambito dello sviluppo software la gestione della configurazione si occupa di 4 attività:
\begin{itemize}
	\item \textbf{Identificazione di configurazione} ovvero dividere il prodotto in configuration item;
	\item \textbf{Controllo di baseline} ovvero definire le baseline che portano ad una milestone garantendone riproducibilità, tracciabilità, analisi e confronto;
	\item \textbf{Gestione delle modifiche} ovvero le richieste di modifiche di utenti, sviluppatore e competizioni sono sottoposte ad analisi, decisione, realizzazione e verifica, e le modifiche devono essere tracciabili e ripristinabili;
\item \textbf{Controllo di versione} cioè VERSIONAMENTO;
	\end{itemize}
Una \textbf{versione} è un'istanza di un determinato configuration item, diversa dalla precedente. Il versionamento consente tramite repository di contenere tutti i configuration item di ogni baseline e la loro storia.
Dunque il versionamento si occupa dei CI di una baseline, e di come vengono identificati e gestiti, mentre la configurazione si occupa anche dell'organizzazione del prodotto, oltre che del versionamento, e dell'integrazione delle parti nel prodotto finale.