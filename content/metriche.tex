\section{Presentare due metriche significative per la misurazione di qualità della progettazione software e del codice( quindi almeno una metrica per ciascun oggetto). Giustificare la scelta in base all'esperienza maturata nell'ambito del proprio progetto didattico. Discuter brevemente l'esito osservato del'eventuale uso pratico di tali metriche}

Lo IEEE definisce una metriche come una misura quantitativa del grado in cui un sistema, componente o processo possiede un certo attributo.
Una metrica per la progettazione software è \textbf{l'instabilità}, che indica il rapporto tra coesione e accoppiamento di una componente. Una \textbf{coesione} indica quanto le parti interne della componente siano legate tra loro. Una coesione alta è indice di una componente modulare, compatta e specializzata. \textbf{L'accoppiamento} indica invece quante dipendenze la componente ha con l'esterno. Maggiore è l'accoppiamento, minore è la mantenibilità e la modularità della componente. Un valore basso di instabilità indica una forte coesione e uno scarso accoppiamento, mentre un valore alto è sintomo di un accoppiamento troppo forte. L'obiettivo di tale metrica è di rendere il software più modulare è manutenibile possibile.

Una metrica per il codice è la\textbf{ complessità ciclomatica}, che indica il numero di cammini indipendenti che l'esecuzione di un metodo può intraprendere. Un valore alto è sintomo di un metodo troppo complesso, scarsamente modulare e manutenibile. 

Nel progetto didattico è stata utilizzata questa metrica che ci ha permesso di rendere i metodi più modulari e facili da testare.