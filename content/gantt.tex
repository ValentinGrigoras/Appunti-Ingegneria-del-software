\section{Fornire una definizione del formalismo noto come "diagramma di Gantt", discuterne concisamentele finalità e modalità d'uso, l'efficacia e i punti deboli eventualmente rilevati nell'esperienza delprogetto didattico}
Il diagramma di Gantt, usato principalmente nelle attività di project management, è uno strumento che serve a pianificare un insieme di attività in un certo periodo di tempo. È costituito da 2 assi. Sull'asse orizzontale si indica il tempo totale del progetto, suddiviso in fasi incrementali (ad esempio, giorni, settimane, mesi), mentre sul asse verticale ci sono le attività da svolgere avente un tempo d’inizio e un tempo di fine, ma non la quantità di lavoro in termini di ore. 

Le barre orizzontali di lunghezza variabile rappresentano le sequenze, la durata e l'arco temporale di ogni singola attività del progetto. Le attività da svolgere possono essere  sovrapposte durante il medesimo arco temporale ad indicare la possibilità dello svolgimento in parallelo di alcune delle attività, oppure dipendenti se un’attività finisce prima che inizi la successiva attività.

Una linea verticale è utilizzata per indicare la data di riferimento. 

Il diagramma di gantt permette quindi di visualizzare chiaramente il flusso di lavoro mostrando la data di inizio e di fine di una determinata attività, consentendo un uso intelligente ed efficace delle risorse.
