\subsection{Spiegare concisamente (dunque a livello di sostanza) la differenza tra il modello di sviluppo iterativo e quello incrementale. Alla luce dell'esperienza acquisita nel progetto didattico, indicare spiegando a posteriori, quale dei due sarebbe stato più adatto al caso}

Nello \textbf{sviluppo iterativo} lo sviluppo del software è organizzato in una serie di mini-progetti brevi, di lunghezza fissa (ad es., 2-4 settimane) chiamati iterazioni. \textbf{Un'iterazione} consiste nella ripetizione di un dato insieme di attività fino a che queste non convergono ad un dato obiettivo, rimandando alla fine l’integrazione delle componenti sviluppate. Ciascuna iterazione comprende le proprie attività di analisi dei requisiti, analisi, progettazione, implementazione, verifica.  Il sistema cresce in modo incrementale da un'iterazione alla successiva, adattandosi ai requisiti, in modo evolutivo, sulla base del feedback delle iterazioni precedenti. Il risultato di ciascuna iterazione è normalmente un sistema incompleto, che converge verso un sistema completo dopo varie iterazioni.
L'articolazione di un progetto iterativo è guidata non da una rigida seguenza di fasi predefinite, ma da una gestione sistematica dei rischi di progetto, per arrivare alla loro progressiva diminuzione.


Nel \textbf{modello incrementale} i cicli non sono più iterazioni ma incrementi. Il termine \textbf{incremento} designa un'aggiunta o un'avanzamento. Ogni incremento attraversa tutte le fasi del modello sequenziale, dall'analisi alla verfica.

Il modello prevede rilasci multipli realizzando un incremento di funzionalità e avvicinandosi sempre più alle attese. Un grande vantaggio è che le funzionalità più importanti vengono trattate per prime; così facendo, queste vengono verficate più volte (dato che ogni ciclo prevede la verfica del software). 

Ogni incremento ha il vantaggio di ridurre il rischio di fallimento, con un approccio più realistico e predisposto ai cambiamenti. Difatti, mentre il modello sequenziale segue un approccio predittivo (cioè basato su piani che devono essere rispettati), il modello incrementale segue un approccio adattativo, dove la realtà è considerata imprevedibile. 

Un grande vantaggio offerto è rappresentato dal fatto che le funzionalità critiche vengono trattate per prime, subendo così una ripetuta verifica.
Valutando il progetto a posteriori, il modello più adatto al caso è probabilmente quello incrementale, che grazie al suo focus sulla verifica delle funzionalità critiche dà maggiori garanzia di corretta implementazione delle stesse. La scelta del modello incrementale permette inoltre di ripetere più e più volte varie fasi del progetto, consentendo ad un team inesperto di impratichirsi maggiormente con le attività ad esse legate.