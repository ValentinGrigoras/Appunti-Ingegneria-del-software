\subsection{Dare una definizione ben fondata del concetto di "architettura software". In relazione a tale concetto, dare una definizione ai termini "framework" e "design pattern" spiegando come questi si integrino fra loro E all'interno di una architettura. }

\textbf{Un'architettura software} è un insieme di elementi architetturali, utilizzati secondo una particolare forma(intesa come organizzazione e strutturazione) insieme a una giustificazione logica che coglie la motivazione per la scelta degli elementi e della forma. Per  \textbf{forma} si intende la divisione di tale sistema in componenti, nella disposizione di essi e nei modi in cui tali componenti comunicano tra loro. \textbf{La giustificazione logica} ha lo scopo di rendere esplicite le motivazioni per la scelta degli elementi e della forma – in particolare, con riferimento al modo in cui questa scelta consente di soddisfare i requisiti/interessi del sistema.
\textbf{Lo scopo dell'architettura software} è di facilitare lo sviluppo, la distribuzione, il funzionamento e la manutenzione del sistema software in esso contenuto, quindi di supportare il ciclo di vita del sistema. \\

\textbf{Un pattern software} è una soluzione provata e ampiamente applicabile a un particolare problema di progettazione che è descritta in una forma standard, in modo che possa essere facilmente condivisa e riusata. 

Un \textbf{design pattern} è la descrizione di oggetti e classi che comunicano tra di loro, personalizzati per risolvere un problema generale di progettazione in un contesto particolare. I design pattern sono spesso utili nel descrivere le connessioni tra elementi architetturali indicando un approccio uniforme nella loro realizzazione.
\textbf{I benefici dei design pattern} soprattutto dal punto di vista dell’architettura del software sono la riduzione del rischio, sulla base di soluzioni provate e ben comprese e un maggior incremento della produttività, della standardizzazione e della qualità. \\

Con la parola \textbf{framework} intendiamo una micro architettura che mette a disposizione tipi estendibili nell'ambito di uno specifico dominio. Nell'archietttura software un framework è una parte di software riutilizzabile ed estendibile.