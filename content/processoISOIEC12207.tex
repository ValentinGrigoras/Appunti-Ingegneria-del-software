\subsection{Fissando l'attenzione sulla definizione di "processo" associata allo standard ISO/IEC 12207, indicare come (secondo quali regole), quando (in quali fasi di progetto) e perché (attraverso quali attività) la vostra esperienza di progetto didattico ha visto attuato tale concetto.}

Un processo è un insieme di attività \textbf{correlate}(ovvero contiene solo attività che hanno a che fare con il progetto)  e \textbf{coese}(un insieme di cose è coeso se tutto ciò che c’è serve, ci deve essere e se non ci fosse mancherebbe, quindi non c'è nulla di superfluo) con l'obiettivo di rispondere ai bisogni in ingresso restituendo risposte (prodotto delle attività del processo) in uscita agendo secondo regole date consumando risorse nel farlo. Significa che un processo non è mai solo perché sopra di lui c’è un \textit{controllo} che sa come le cose stanno andando perché emette vincoli sul modo in cui il processo lavora misurando l'efficienza e l'efficacia.

L'efficienza si misura con l’efficienza produttiva. Guardo quindi il rapporto tra quantità di prodotto realizzato e risorse utilizzate.

L'efficacia invece si misura in base a quanti obiettivi interni (del fornitore) o esterni (gradimento del cliente) raggiungo.

L’insieme di efficienza ed efficacia si chiama economicità, ovvero raggiungo gli obiettivi se sono efficace consumando poche risorse.

L'ISO/IEC 12207 contiene una descrizione approfondita dei processi del ciclo di vita del software, ed è infatti il modello più noto e riferito, anche se ne esistono altri. Questo modello è ad alt livello, ed identifica i processi dello sviluppo software, ed ha una struttura modulare che permette, nel processo di specializzazione, di identificare le entità responsabili dei processi ed i prodotti dei processi.
Secondo questo modello si hanno processi (processes), che sono divisi in attività (activities) che, a loro volta, sono divisi in compiti (tasks). Così si ha una struttura modulare (perchè i processi si interfacciano), ma con una forte coesione (perchè i compiti sono chiusi). I processi descritti in ISO 12207  hanno lo scopo di eliminare tutti gli sprechi di tempo e risorse, eliminando le particolari attività per un progetto specifico.
Nel corso del progetto didattico, si è innanzitutto cercato di determinare i processi necessari allo sviluppo del prodotto attenendosi fedelmente allo standard ISO/IEC 12207. Si è poi passati alla definizione generica delle attività costituenti ogni processo per poi contestualizzarla e dettagliarla sempre più con il progredire del progetto. Si è dunque assicurati che all’istanziazione di ogni processo questo fosse normato e ben definito rispetto alle attività che lo compongono e alle sue pre e post condizioni. Da un punto di vista pratico, questo approccio ha permesso di lavorare secondo una migliore suddivisione in task, migliorarndo progressivamente la qualità dei processi una volta concluse le relative attività e verificati i relativi prodotti.