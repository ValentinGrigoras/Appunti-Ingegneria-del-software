\subsection{Fornire una definzione dei concetti di \textbf{milestone} e \textbf{baseline}, indicando come ciascuno di essi sia da utilizzare all'interno delle attività di progetto.}
Una \textbf{baseline} è un punto di avanzamento consolidato, fissato strategicamente in risposta a qualche discussione con gli stakeholders. Una baseline è fatta di parti chiamati CI (configuration item ) utili al raggiungimento di obiettivi strategici in un tempo breve. Le parti di cui è fatta una baseline (le quali hanno un numero di versioni “as many as needed”), esistono perchè assolvono un obiettivo.
Ogni punto di avanzamento viene fissato precedentemente in modo strategico dalla best practice, ma il numero di baseline non è deciso a priori:  solo gli obiettivi sono decisi a priori.
Una baseline si costruisce con la configurazione e si mantiene con il versionamento.

Una \textbf{milestone} definisce un traguardo da raggiungere, pianificato e misurabile. Un traguardo è la fine di una fase di progetto in corrispondenza della quale può essere rivisto l'avanzamento del lavoro. Ogni milestone deve essere documentata da un breve report che riassume lo stato del software che si sta sviluppando e ne verifica la completezza rispetto a quanto specificato nel piano di progetto.
Questo report chiamato output formale della milestone viene poi presentato al responsabile.
Durante le fasi critiche del progetto come per esempio nella fase di progettazione, vengono fatte delle consegne al cliente che attestastano lo stato di avanzamento del prodotto. Le consegne in genere sono delle milestone, ma le milestone non sempre sono delle consegne, in quanto una milestone può rappresentare dei risultati interni usati dal responsabile del progetto per verificare i progressi interni; questo tipo di milestone non vengono consegnati al cliente. Una milestone è concretizzata da almeno una baseline e dev'essere specifica, raggiungibile, misurabile (per quantità di impegno necessario), traducibile in compiti assegnabili e dimostrabile agli stakeholder. Il numero di milestone lo decide il fornitore.


Nel progetto didattico, le milestone, sono una parte fondamentale nella pianificazione del progetto perchè tramite l'utilizzo del diagramma di gantt riusciamo a tener traccia delle attività svolte ad ogni revisione. Ogni milestone viene segnalata nel piano di progetto da un rombo che sta a indicare la data della verifica di una certa milestone. Le milestone che il gruppo ha seguito sono quelle corrispondenti alle varie revisioni e ai vari incontri con il professor Cardin.
Utilizzare le milestione in un progetto software aiuta a tener traccia del tempo che si sta dedicando ad ogni attività e a monitorare le varie scadenze associate alle milestone.